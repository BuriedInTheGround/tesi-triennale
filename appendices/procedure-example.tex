\chapter{Procedura di codifica: esempio numerico}

\thispagestyle{empty}

In questa appendice, riportiamo un esempio numerico della procedura di codifica
proposta nella Sezione~\ref{sec:procedure}. Nella prima parte, effettuiamo
anche l'operazione di sparsificazione descritta nella
Sottosezione~\ref{subsec:p-matrix-sparsification}.
%\newline
Si noti che i valori numerici utilizzati hanno il solo scopo di illustrare la
procedura proposta e sono pertanto artificiosi.

\medskip

Sia \(\set{V}\), tale che \(\abs{\set{V}} = \gamma = 2^{b_\mathrm{U}}\),
l'alfabeto della sorgente associata al canale di UL, equivalente al codebook
utilizzato per la rappresentazione locale della versione quantizzata del canale
di UL stesso alla BS. Similmente, sia \(\set{D}\), tale che \(\abs{\set{D}} =
\beta = 2^{b_\mathrm{B}}\), l'alfabeto della sorgente associata al canale di
DL, equivalente al codebook utilizzato per la rappresentazione della versione
quantizzata del canale di DL.

Sia \(b_\mathrm{B} = b_\mathrm{U} = 2\), da cui \(\beta = \gamma = 4\). Sia
data la seguente matrice \(\bm{P} \in [0,1]^{4 \times 4}\) di densità di
probabilità discreta congiunta.
\begin{equation}
    \bm{P} =
    \begin{pmatrix}
        \frac{1}{36} & \frac{1}{72} & \frac{1}{72} & \frac{1}{18} \\
        \frac{1}{8}  & \frac{1}{36} & \frac{1}{18} & \frac{1}{8}  \\
             0       & \frac{1}{72} & \frac{1}{48} & \frac{1}{48} \\
        \frac{1}{72} & \frac{1}{36} & \frac{7}{16} & \frac{1}{48}
    \end{pmatrix} \label{eq:apx-p-matrix}
    %\approx
    %\begin{pmatrix}
    %    0.02777 & 0.01388 & 0.01388 & 0.05555 \\
    %    0.125   & 0.02777 & 0.05555 & 0.125   \\
    %    0       & 0.01388 & 0.02083 & 0.02083 \\
    %    0.01388 & 0.02777 & 0.4375  & 0.02083
    %\end{pmatrix}
\end{equation}
Pertanto, le densità di probabilità marginali delle sorgenti associate ai
canali di UL e DL sono quelle riassunte nella seguente tabella.
\begin{table}[ht]
    \centering
    \renewcommand{\arraystretch}{1.4}
    \begin{tabular}{lcccc}
        \toprule
        \(i\) & 1 & 2 & 3 & 4 \\
        \midrule
        \(p(\bm{v}_i)\) & \(\frac{1}{9}\) & \(\frac{1}{3}\)  & \(\frac{1}{18}\)  & \(\frac{1}{2}\) \\
        \(p(\bm{d}_i)\) & \(\frac{1}{6}\) & \(\frac{1}{12}\) & \(\frac{19}{36}\) & \(\frac{2}{9}\) \\
        \bottomrule
    \end{tabular}
    \caption{Densità di probabilità marginali di $\set{V}$ e $\set{D}$.}
    \label{tab:apx-marginal-probability}
\end{table}
\newline
Applichiamo la procedura di sparsificazione (vedi
Sottosezione~\ref{subsec:p-matrix-sparsification}) alla matrice $\bm{P}$ in
\eqref{eq:apx-p-matrix}, scegliendo arbitrariamente di mantenere solo \(3\)
elementi non nulli, ottenendo quindi la seguente matrice \(\bm{\breve{P}}\)
sparsa.
\begin{equation}
    \bm{\breve{P}} =
    \begin{pmatrix}
             0       & 0 &      0       &      0       \\
        \frac{1}{8}  & 0 &      0       & \frac{1}{8}  \\
             0       & 0 &      0       &      0       \\
             0       & 0 & \frac{7}{16} &      0
    \end{pmatrix} \label{eq:apx-p-matrix-sparse}
\end{equation}
Facendo riferimento alla matrice \(\bm{\breve{P}}\) appena ottenuta, costruiamo
i sottoinsiemi di \(\set{V}\) descritti nella fase~\ref{itm:subsets-of-v} della
procedura a pagina~\pageref{itm:subsets-of-v}. Si noti che, poiché la matrice è
molto sparsa, non è inusuale ottenere sottoinsiemi vuoti, come in questo caso
con \(\set{V}_2\).
\begin{table}[ht]
    \centering
    \renewcommand{\arraystretch}{1.4}
    \begin{tabular}{lll}
        \toprule
        \(j\) & Simbolo in \(\set{D}\) & Sottoinsieme di \(\set{V}\) \\
        \midrule
        \(1\) & \(\bm{d}_1\) & \(\set{V}_1 = \{ \bm{v}_2 \}\) \\
        \(2\) & \(\bm{d}_2\) & \(\set{V}_2 = \emptyset\)      \\
        \(3\) & \(\bm{d}_3\) & \(\set{V}_3 = \{ \bm{v}_4 \}\) \\
        \(4\) & \(\bm{d}_4\) & \(\set{V}_4 = \{ \bm{v}_2 \}\) \\
        \bottomrule
    \end{tabular}
    \caption{Simboli in $\set{D}$ e i corrispondenti sottoinsiemi di
    $\set{V}$.}
    \label{tab:apx-subsets-of-v}
\end{table}
\newline
Determiniamo ora le possibili partizioni di \(\set{D}\), utilizzando i $\beta =
4$ sottoinsiemi di \(\set{V}\) riportati in Tabella~\ref{tab:apx-subsets-of-v},
come indicato nella fase~\ref{itm:partitions-of-d} della procedura. Ricordiamo
che, detto \(B_n\) il numero di Bell di indice \(n\), la condizione
\eqref{eq:partitions-condition} porta il numero di partizioni possibili, così
come considerate nella procedura, a essere \(\le B_\beta\) (tipicamente vale il
minore stretto). Calcoliamo inoltre, per ognuna delle partizioni trovate, la
relativa entropia, associando ad ogni sottoinsieme un simbolo con probabilità
pari alla somma delle probabilità dei suoi elementi, così come descritto nella
fase~\ref{itm:partitions-entropy}. Riportiamo i risultati ottenuti in
Tabella~\ref{tab:apx-partitions-of-d}.
\begin{table}[ht]
    \centering
    \renewcommand{\arraystretch}{1.4}
    \begin{tabular}{lll}
        \toprule
        n. & Partizione di \(\set{D}\) & Entropia \([\mathrm{bit}]\)\\
        \midrule
        1 & \(\{ \{\bm{d}_1\}, \{\bm{d}_2\}, \{\bm{d}_3\}, \{\bm{d}_4\} \}\) & 1.69838 \\
        2 & \(\{ \{\bm{d}_1\}, \{\bm{d}_2\}, \{\bm{d}_3, \bm{d}_4\} \}\) & 1.04085 \\
        3 & \(\{ \{\bm{d}_1\}, \{\bm{d}_2, \bm{d}_4\}, \{\bm{d}_3\} \}\) & 1.44008 \\
        4 & \(\{ \{\bm{d}_1\}, \{\bm{d}_2, \bm{d}_3\}, \{\bm{d}_4\} \}\) & 1.34722 \\
        5 & \(\{ \{\bm{d}_1\}, \{\bm{d}_2, \bm{d}_3, \bm{d}_4\} \}\) & 0.65002 \\
        6 & \(\{ \{\bm{d}_1, \bm{d}_3\}, \{\bm{d}_2\}, \{\bm{d}_4\} \}\) & 1.14627 \\
        7 & \(\{ \{\bm{d}_1, \bm{d}_3\}, \{\bm{d}_2, \bm{d}_4\} \}\) & 0.88797 \\
        8 & \(\{ \{\bm{d}_1, \bm{d}_2\}, \{\bm{d}_3\}, \{\bm{d}_4\} \}\) & 1.46881 \\
        9 & \(\{ \{\bm{d}_1, \bm{d}_2\}, \{\bm{d}_3, \bm{d}_4\} \}\) & 0.81127 \\
        10 & \(\{ \{\bm{d}_1, \bm{d}_2, \bm{d}_3\}, \{\bm{d}_4\} \}\) & 0.76420 \\
        \bottomrule
    \end{tabular}
    \caption{Partizioni di $\set{D}$ valide, e relativi valori di entropia in
    $\mathrm{bit}$.}
    \label{tab:apx-partitions-of-d}
\end{table}
\newline
Osservando i risultati ottenuti, il minimo valore di entropia è dato dalla
quinta partizione, e di conseguenza la partizione di \(\set{D}\) che lo UT
dovrà utilizzare è \(\set{P}_\mathrm{min} = \{ \{\bm{d}_1\}, \{\bm{d}_2,
\bm{d}_3, \bm{d}_4\} \}\), con una relativa entropia di
$0.65002\,\mathrm{bit}$.

%%%$ Vogliamo far notare come, dal punto di vista grafico, osservando la
%%%$ Figura~\ref{fig:subsets-of-v}, la condizione \eqref{eq:partitions-condition}
%%%$ che determina se una partizione è valida o meno esprime quanto segue: se due
%%%$ sottoinsiemi di $\set{V}$ tra quelli individuati si intersecano (come
%%%$ $\set{V}_1$ e $\set{V}_4$ in questo caso), i corrispondenti simboli in
%%%$ \begin{figure}[ht]
%%%$     \centering
%%%$     \input{appendices/procedure-example/subsets-of-v.tex}
%%%$     %\caption{Rappresentazione insiemistica dell'insieme $\set{V}$ e dei suoi
%%%$     %sottoinsiemi così come ricavati nella fase~\ref{itm:subsets-of-v} della
%%%$     %procedura.}
%%%$     \caption{Rappresentazione insiemistica di $\set{V}$ e dei suoi
%%%$     sottoinsiemi.}
%%%$     \label{fig:subsets-of-v}
%%%$ \end{figure}
%%%$ $\set{D}$ (qui $\bm{d}_1$ e $\bm{d}_4$) non potranno mai stare all'interno di
%%%$ uno stesso insieme di una partizione di $\set{D}$. È possibile verificare
%%%$ quanto appena descritto esaminando le partizioni riportate nella tabella
%%%$ soprastante. La non sovrapposizione tra sottoinsiemi di $\set{V}$ è esattamente
%%%$ ciò che permette alla BS di ricavare correttamente il canale di DL a partire
%%%$ dal solo indice della partizione selezionata dallo UT.

%%%$ Poiché c'è una corrispondenza biunivoca tra i simboli in $\set{D}$ e i
%%%$ sottoinsiemi di $\set{V}$ determinati dalla fase~\ref{itm:subsets-of-v}...

Si noti come l'utilizzo della partizione di \(\set{D}\) selezionata fornisca
una riduzione del tasso di codifica del \(50\%\), da \(2\;\mathrm{bit}\) a
\(1\;\mathrm{bit}\): è infatti sufficiente per lo UT inviare alla BS l'indice
$h \in \{1,2\}$ di uno dei due sottoinsiemi della partizione ($\set{D}_1 =
\{\bm{d}_1\}$ e $\set{D}_2 = \{\bm{d}_2, \bm{d}_3, \bm{d}_4\}$), anziché
l'indice di uno dei quattro simboli in \(\set{D}\).

% \(\{ \{\bm{d}_1\}, \{\bm{d}_2\}, \{\bm{d}_3\}, \{\bm{d}_4\} \}\) & 1.6983893 \\
% \(\{ \{\bm{d}_1\}, \{\bm{d}_2\}, \{\bm{d}_3, \bm{d}_4\} \}\) & 1.040852083 \\
% \(\{ \{\bm{d}_1\}, \{\bm{d}_2, \bm{d}_4\}, \{\bm{d}_3\} \}\) & 1.440087625 \\
% \(\{ \{\bm{d}_1\}, \{\bm{d}_2, \bm{d}_3\}, \{\bm{d}_4\} \}\) & 1.34722304 \\
% \(\{ \{\bm{d}_1\}, \{\bm{d}_2, \bm{d}_3, \bm{d}_4\} \}\) & 0.6500224216 \\
% \(\{ \{\bm{d}_1, \bm{d}_3\}, \{\bm{d}_2\}, \{\bm{d}_4\} \}\) & 1.146277995 \\
% \(\{ \{\bm{d}_1, \bm{d}_3\}, \{\bm{d}_2, \bm{d}_4\} \}\) & 0.8879763195 \\
% \(\{ \{\bm{d}_1, \bm{d}_2\}, \{\bm{d}_3\}, \{\bm{d}_4\} \}\) & 1.468815341 \\
% \(\{ \{\bm{d}_1, \bm{d}_2\}, \{\bm{d}_3, \bm{d}_4\} \}\) & 0.8112781245 \\
% \(\{ \{\bm{d}_1, \bm{d}_2, \bm{d}_3\}, \{\bm{d}_4\} \}\) & 0.7642045065 \\
