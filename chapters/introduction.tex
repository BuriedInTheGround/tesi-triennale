\chapter{Introduzione}

\thispagestyle{empty}

\vspace{9.1pt}  % Fix the underfull warning.

Le prestazioni di un sistema di comunicazioni possono essere migliorate se il
trasmettitore conosce le caratteristiche del canale sul quale avviene la
trasmissione. Ad esempio, quando più canali \textit{additive white Gaussian
noise} (\textit{AWGN}, rumore Gaussiano bianco additivo) sono disponibili in
parallelo, la conoscenza del canale consiste nel sapere l'attenuazione e la
differenza di fase introdotta in ogni canale. Le caratteristiche del canale
sono di solito acquisite con un processo di stima che opera sul segnale
ricevuto, pertanto la stima è disponibile tipicamente al ricevitore.

L'ottenimento di una stima del canale al trasmettitore dipende invece da come i
dispositivi alternano la trasmissione e la ricezione dei segnali e da quali
canali usano nelle due fasi. Nelle reti cellulari, due modalità sono possibili,
vale a dire, \textit{time-division duplexing} (\textit{TDD}, duplex a divisione
di tempo) e \textit{frequency-division duplexing} (\textit{FDD}, duplex a
divisione di frequenza). In modalità TDD, le trasmissioni in \textit{downlink}
(\textit{DL}) (dalla \textit{base station} (\textit{BS}, stazione base) agli
\textit{user terminals} (\textit{UT}s, terminali utente)) e le trasmissioni in
\textit{uplink} (\textit{UL}) (dagli UTs alla BS) avvengono sulla stessa banda
di frequenza, ma in istanti di tempo differenti. In modalità FDD, la BS e gli
UTs utilizzano bande di frequenza separate per lo scambio di dati in UL e DL,
consentendo la trasmissione simultanea in entrambe le direzioni. Ottenere la
stima delle caratteristiche del canale al trasmettitore risulta più semplice
quando si opera in modalità TDD a causa della reciprocità del canale. In questo
caso infatti quando la BS ha stimato il canale in ricezione dallo UT, conosce
anche il canale in trasmissione verso lo UT. D'altra parte, operando in
modalità FDD, lo UT stima il canale e invia la stima alla BS. Nel seguito
faremo riferimento a questo procedimento come approccio feedback-only. In
particolare, la BS trasmette dei segnali pilota tramite i quali lo UT stima il
canale e ne manda indietro una versione quantizzata alla BS.

In questa trattazione, proponiamo una procedura di feedback del CSI a tasso
variabile con approccio feedback-only, il cui obiettivo è di minimizzare
l'overhead della segnalazione. Per fare ciò, sfrutteremo la dipendenza
statistica presente tra il canale di UL e il canale di DL, che origina
dall'ambiente fisico in cui si trova il canale fisico di comunicazione.

Il materiale riportato in questa tesi riprende ed estende quanto presentato in
\cite{https://doi.org/10.1002/ett.3628}.

\subsection*{Notazione}

Presentiamo ora la notazione che verrà utilizzata nel seguito in questo
documento.

Sia \(\C\) l'insieme dei numeri complessi. Indichiamo vettori riga e matrici in
grassetto, rispettivamente con lettere minuscole e maiuscole.  Pertanto,
\(x_i\) indica l'\(i\)-esimo elemento del vettore \(\bm{x}\), mentre
\(A_{i,j}\) denota l'elemento posizionato sull'\(i\)-esima colonna e la
\(j\)-esima riga della matrice \(\bm{A}\). Denotiamo il trasposto della matrice
\(\bm{A}\) con \(\tran{A}\) e il trasposto coniugato con \(\conjtran{A}\).  Il
prodotto interno tra due vettori \(\bm{x},\bm{y} \in \C^{1 \times N}\) è dato
da \(\bm{x}\conjtran{y} \define \sum_i x_i \conj{y_i}\), con \(\conj{y_i}\) il
complesso coniugato di \(y_i\). La norma di un vettore \(\bm{x}\) è definita
come \(\lnorm \bm{x} \rnorm = \lnorm \bm{x} \rnorm_2 \define
\mleft(\bm{x}\conjtran{x}\mright)^{\sfrac{1}{2}}\). Gli insiemi sono denotati
con lettere maiuscole calligrafiche, e \(\abs{\set{A}}\) indica la cardinalità
dell'insieme \(\set{A}\). Infine, denotiamo la probabilità di un evento \(E\)
con \(\prob{E}\), il valore atteso di una variabile casuale \(X\) con
\(\expect{X}\), e l'entropia di \(X\) con \(\entropy{X}\).

