\chapter{Introduzione}

\thispagestyle{empty}

\vspace{9.6pt}  % Fix the underfull warning.

Nei sistemi cellulare, il \textit{channel-state information} (\textit{CSI},
informazione sullo stato del canale) fa riferimento a note proprietà del canale
di comunicazione. La stima del canale (in inglese, ``channel estimation''),
ovvero il metodo usato per stimare il CSI, è di fondamentale importanza per
massimizzare le prestazioni dei ricevitori nel compensare le alterazioni
introdotte dal canale wireless.

L'ottenimento di una stima accurata del canale dipende dalla modalità operativa
del canale, vale a dire, \textit{time-division duplexing} (\textit{TDD}, duplex
a divisione di tempo) e \textit{frequency-division duplexing} (\textit{FDD},
duplex a divisione di frequenza). In modalità TDD, le trasmissioni in
\textit{downlink} (\textit{DL}) (dalla \textit{base station} (\textit{BS},
stazione base) agli \textit{user terminals} (\textit{UT}s, terminali utente)) e
le trasmissioni in \textit{uplink} (\textit{UL}) (dagli UTs alla BS) avvengono
sulla stessa banda di frequenza, ma in istanti di tempo differenti. In modalità
FDD, la BS e gli UTs utilizzano bande di frequenza separate per lo scambio di
dati in UL e DL, consentendo la trasmissione simultanea in entrambe le
direzioni. La stima del canale risulta più semplice quando si opera in modalità
TDD a causa della reciprocità del canale. D'altra parte, operando in modalità
FDD, la BS solitamente ottiene una stima del canale di DL dallo UT con cui la
avviene la comunicazione; nel seguito faremo riferimento a questo procedimento
come approccio feedback-only.  In particolare, la BS trasmette dei segnali
pilota tramite i quali lo UT stima il canale e ne manda indietro una versione
quantizzata alla BS.

In questa trattazione, proponiamo una procedura di feedback del CSI a tasso
variabile con approccio feedback-only, il cui obiettivo è di minimizzare
l'overhead della segnalazione. Per fare ciò, sfrutteremo la dipendenza
statistica presente tra il canale di UL e il canale di DL, che origina
dall'ambiente fisico in cui si trova il canale fisico di comunicazione.

\subsection*{Notazione}

Presentiamo ora la notazione che verrà utilizzata nel seguito in questo
documento.

Sia \(\C\) l'insieme dei numeri complessi. Indichiamo vettori riga e matrici in
grassetto, rispettivamente con lettere minuscole e maiuscole.  Pertanto,
\(x_i\) indica l'\(i\)-esimo elemento del vettore \(\bm{x}\), mentre
\(A_{i,j}\) denota l'elemento posizionato sull'\(i\)-esima colonna e la
\(j\)-esima riga della matrice \(\bm{A}\). Denotiamo il trasposto della matrice
\(\bm{A}\) con \(\tran{A}\) e il trasposto coniugato con \(\conjtran{A}\).  Il
prodotto interno tra due vettori \(\bm{x},\bm{y} \in \C^{1 \times N}\) è dato
da \(\bm{x}\conjtran{y} \define \sum_i x_i \conj{y_i}\), con \(\conj{y_i}\) il
complesso coniugato di \(y_i\). La norma di un vettore \(\bm{x}\) è definita
come \(\lnorm \bm{x} \rnorm = \lnorm \bm{x} \rnorm_2 \define
\mleft(\bm{x}\conjtran{x}\mright)^{\sfrac{1}{2}}\). Gli insiemi sono denotati
con lettere maiuscole calligrafiche, e \(\abs{\set{A}}\) indica la cardinalità
dell'insieme \(\set{A}\). Infine, denotiamo la probabilità di un evento \(E\)
con \(\prob{E}\), il valore atteso di una variabile casuale \(X\) con
\(\expect{X}\), e l'entropia di \(X\) con \(\entropy{X}\).

