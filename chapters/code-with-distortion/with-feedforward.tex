\section{Codifica con feedforward}
\label{sec:with-feedforward}

Nel caso in cui si permette una distorsione nella ricostruzione del canale alla
BS, la trasmissione di un segnale di feedforward può ridurre il tasso di
feedback. Si noti che questo non avveniva nel caso senza distorsione, come
dimostrato nella Sezione~\ref{sec:feedback-only}.

Supponiamo in particolare che la BS trasmetta allo UT una descrizione di
\(\bm{h}^\mathrm{(U)}\) senza distorsione, quindi che in fase di codifica lo UT
conosca perfettamente sia \(\bm{h}^\mathrm{(D)}\) che \(\bm{h}^\mathrm{(U)}\).
In questo caso abbiamo un problema di teoria della distorsione condizionata
(\textit{Conditional Rate-Distortion Theory}). Usando le definizioni introdotte
nella sezione precedente, in questo caso si ha \cite{Gray1972ConditionalRT} che
\begin{equation}
    b_\mathrm{B} \ge \inf_{p(
        \hat{\bm{h}}^\mathrm{(D)} \mid \bm{h}^\mathrm{(D)}, \bm{h}^\mathrm{(U)}
    )}
    \mutual{
        \bm{h}^\mathrm{(D)}
    }{
        \hat{\bm{h}}^\mathrm{(D)} \given \bm{h}^\mathrm{(U)}
    } , \label{eq:rate-with-wz-and-feedforward}
\end{equation}
dove \(\hat{\bm{h}}^\mathrm{(D)}\) è la versione distorta di
\(\bm{h}^\mathrm{(D)}\) ottenuta al decodificatore, ovvero alla base station.

Usando ipotesi più stringenti, supponiamo ora che la BS trasmetta allo UT una
descrizione del canale di UL quantizzata \(q(\bm{h}^\mathrm{(U)})\). Questa
condizione corrisponde al caso reale, infatti la quantizzazione è necessaria
essendo \(\bm{h}^\mathrm{(U)}\) a valori continui. In questo caso la
rappresentazione del canale di UL allo UT è imperfetta, e abbiamo allora un
problema di distorsione con informazione laterale mista
(\textit{Rate-Distortion With Mixed Types of Side Information}) e si ha
\cite{1614094} che
\begin{equation}
    b_\mathrm{B} \ge \inf_{W \in \set{M}_{
        \bm{h}^\mathrm{(D)} \mid \bm{h}^\mathrm{(U)} \{Z\}
    }(p, D)} \mutual{
        \bm{h}^\mathrm{(D)}
    }{
        W \given \bm{h}^\mathrm{(U)} \:,\; Z
    } , \label{eq:rate-with-wz-mixed-si}
\end{equation}
dove \(\set{M}_{\bm{h}^\mathrm{(D)} \mid \bm{h}^\mathrm{(U)} \{Z\}}(p, D)\) è
l'insieme di tutte le variabili aleatorie \(W\) descritte da un canale di test
con la proprietà che \(W \to (\bm{h}^\mathrm{(D)}, \bm{h}^\mathrm{(U)}) \to Z\)
e per le quali esiste una funzione \(f: \set{W} \times \set{U} \times \set{Z}
\to \hat{\set{D}}\) tale che
\begin{equation}
    \expect{D(\bm{h}^\mathrm{(D)}, \hat{\bm{h}^\mathrm{(D)}})} \le d, \quad
    \textrm{dove } \hat{\bm{h}^\mathrm{(D)}} = f(W, \bm{h}^\mathrm{(U)}, Z) .
\end{equation}
