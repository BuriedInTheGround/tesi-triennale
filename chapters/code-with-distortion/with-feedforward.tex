\section{Codifica con feedforward}

Nel caso in cui si permette una distorsione nella ricostruzione del canale alla
BS, la trasmissione di un segnale di feedforward può ridurre il tasso di
feedback. Si noti che questo non avveniva nel caso senza distorsione, come
dimostrato nella Sezione~\ref{sec:feedback-only}.

Supponiamo in particolare che la BS trasmetta allo UT una descrizione di
\(\bm{h}^\mathrm{(U)}\) senza distorsione, quindi che in fase di codifica lo UT
conosca perfettamente sia \(\bm{h}^\mathrm{(D)}\) che \(\bm{h}^\mathrm{(U)}\).
In questo caso abbiamo un problema di teoria della distorsione condizionata
(\textit{Conditional Rate-Distortion Theory}). Usando le definizioni introdotte
nella sezione precedente, in questo caso si ha \cite{Gray1972ConditionalRT} che
\begin{equation}
    b_\mathrm{B} \ge \inf_{p(
        \hat{\bm{h}}^\mathrm{(D)} \mid \bm{h}^\mathrm{(D)}, \bm{h}^\mathrm{(U)}
    )}
    \mutual{
        \bm{h}^\mathrm{(D)}
    }{
        \hat{\bm{h}}^\mathrm{(D)} \given \bm{h}^\mathrm{(U)}
    } , \label{eq:rate-with-wz-and-feedforward}
\end{equation}
dove \(\hat{\bm{h}}^\mathrm{(D)}\) è la versione distorta di
\(\bm{h}^\mathrm{(D)}\) ottenuta al decodificatore, ovvero alla base station.
