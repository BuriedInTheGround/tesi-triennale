\chapter{Conclusioni}

\thispagestyle{empty}

In questa tesi ho riportato e analizzato quanto noto in letteratura sul
problema della stima del canale di DL alla BS per sistemi cellulari FDD. Questo
problema viene affrontato sfruttando la parziale dipendenza tra i canali di UL
e DL data dal mezzo fisico di trasmissione. Sono stati considerati
principalmente tre casi: il primo in assenza di distorsione, basato su una
segnalazione feedback-only a tasso variabile, e gli ultimi due in presenza di
distorsione, di cui uno senza e l'altro con segnalazione di feedforward.

\begin{table}[ht]
    \centering
    \renewcommand{\arraystretch}{1.4}
    \begin{tabular}{lll}
        \toprule
        Distorsione & Bit in feedforward & Bit in feedback \\
        \midrule
        Assente                     & \(b_\mathrm{F} = 0\) & \(b_\mathrm{B} = B\) \\
        Presente                    & \(b_\mathrm{F} = 0\) & \(b_\mathrm{B} > 0\) \eqref{eq:rate-with-wz} \\
        Presente (solo in feedback) & \(b_\mathrm{F} > 0\) & \(b_\mathrm{B} > 0\) \eqref{eq:rate-with-wz-and-feedforward} \\
        Presente                    & \(b_\mathrm{F} > 0\) & \(b_\mathrm{B} > 0\) \eqref{eq:rate-with-wz-mixed-si} \\
        \bottomrule
    \end{tabular}
    \caption{
        Configurazioni di bit ripartiti tra la segnalazione di feedforward e
        feedback nelle differenti situazioni studiate.
    }
    \label{tab:results-summary}
\end{table}

La precedente tabella riassume i risultati ottenuti nelle diverse
configurazioni considerate. In questa tesi un'attenzione particolare è stata
posta sul primo caso, quello in assenza di distorsione, di cui in
Sezione~\ref{sec:procedure} viene riportata una procedura di codifica che
utilizza l'approccio feedback-only, la cui efficacia è stata dimostrata in
\cite{https://doi.org/10.1002/ett.3628}.

Come osservato in Sezione~\ref{sec:with-feedforward}, un ulteriore risultato
presentato in questa tesi è che la presenza della segnalazione di feedforward,
non utile a ridurre il NMSE nel caso senza distorsione, sia invece di aiuto al
fine di ricostruire con maggior affidabilità il canale di DL alla BS nel caso
in cui vi sia distorsione. In particolare, in una situazione realistica, poiché
il canale di UL inviato in feedforward allo UT dev'essere quantizzato, la
condizione al numero di bit per il tasso di feedback che si applica è la
\eqref{eq:rate-with-wz-mixed-si}.
