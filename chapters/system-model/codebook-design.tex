\section{Progettazione dei codebook}

Un codebook di \(\beta\) in \(\C^{1 \times N/2}\) elementi è completamente
definito da
\begin{itemize}

    \item una partizione \(\{\set{R}_j, j=1,\dots,\beta\}\) di
        \(\C^{1 \times N/2}\), tale che \(\bigcup_{j=1}^\beta \set{R}_j = \C^{1
        \times N/2}\), e che \(\set{R}_j \cap \set{R}_\ell = \emptyset\) per
        \(j \neq \ell\);

    \item \(\beta\) vettori \(\bm{w}_1,\dots,\bm{w}_\beta\), dove il vettore
        \(i\)-esimo \(\bm{w}_i\) è associato all'\(i\)-esimo insieme
        \(\set{R}_i\) della partizione descritta al punto precedente.

\end{itemize}

Sia \(\set{U} = \{\bm{u}_1,\dots,\bm{u}_\alpha\}\) il codebook di \(\alpha\)
parole di codice utilizzato alla BS per la quantizzazione della stima del
canale di UL. Sia
\begin{equation}
    c^\mathrm{(F)} = \argmin_{i \in \{1,\dots,\alpha\}}
        \lnorm \bm{u}_i - \bm{h}^\mathrm{(U)} \rnorm^2
\end{equation}
detta \textit{variabile} (o indice) \textit{di feedforward}.

Sia \(\{\set{D}_1,\dots,\set{D}_\alpha\}\) un insieme di \(\alpha\) codebook
noto sia alla BS che allo UT, dove l'\(i\)-esimo codebook \(\set{D}_i =
\{\bm{d}_1^{(i)},\dots,\bm{d}_\beta^{(i)}\}\) contiene \(\beta\) parole di
codice. Sia
\begin{equation}
    c^\mathrm{(B)} = \argmin_{j \in \{1,\dots,\beta\}}
        \lnorm \bm{d}_j^{(c^\mathrm{(F)})} - \bm{h}^\mathrm{(D)} \rnorm^2
\end{equation}
detta \textit{variabile} (o indice) \textit{di feedback}.

Infine, sia \(\set{V} = \{\bm{v}_1,\dots,\bm{v}_\gamma\}\) il
codebook di \(\gamma \ge \alpha\) parole di codice utilizzato alla BS per la
quantizzazione locale del canale di UL.

La procedura di quantizzazione associa ogni vettore \(\bm{x} \in \C^{1 \times
N/2}\) a una regione dello spazio \(\C^{1 \times N/2}\) stesso, ovvero \(\bm{x}
\in \set{R}_{\ihat}\), e pertanto il vettore quantizzato è \(q(\bm{x}) =
\bm{w}_{\ihat}\). L'operazione di progettazione dei codebook, cioè la scelta
delle regioni di quantizzazione \(\{\set{R}_i\}\) e dei relativi vettori
\(\bm{w}_i\), ha l'obiettivo di minimizzare una data metrica di quantizzazione.
Nel nostro caso, i codebook di UL, \(\set{U}\) e \(\set{V}\), sono progettati
con lo scopo di minimizzare il NMSE, vale a dire,
\begin{equation}
    \expect[\bm{h}^\mathrm{(U)}]{
        \lnorm \bm{h}^\mathrm{(U)} - q(\bm{h}^\mathrm{(U)}) \rnorm^2
    }, \label{eq:min-ul-codebook}
\end{equation}
(o l'equivalente per \(\set{V}\)) dove il valore atteso è valutato sulla
distribuzione di \(\bm{h}^\mathrm{(U)}\). Per il codebook di DL, ovvero
\(\set{D}_{c^\mathrm{(F)}}\), invece, la metrica di quantizzazione da
minimizzare è
\begin{equation}
    \expect[\bm{h}^\mathrm{(D)} \mid q(\bm{h}^\mathrm{(U)})]{
        \lnorm \bm{h}^\mathrm{(D)} - q(\bm{h}^\mathrm{(D)}) \rnorm^2
    }, \label{eq:min-dl-codebook}
\end{equation}
dove il valore atteso è valutato sulla distribuzione di
\(\bm{h}^\mathrm{(D)}\), dato che il canale di UL è stato quantizzato come
\(q(\bm{h}^\mathrm{(U)}) = \bm{u}_{c^\mathrm{(F)}}\).

Per la progettazione dei codebook tramite minimizzazione della
\eqref{eq:min-ul-codebook} e \eqref{eq:min-dl-codebook}, poiché nella pratica
le densità di probabilità di \(\bm{h}^\mathrm{(U)}\) e \(\bm{h}^\mathrm{(D)}
\mid q(\bm{h}^\mathrm{(U)})\) non sono note a priori, è necessario fare ricorso
a una procedura online, utilizzando vettori campionati aventi le statistiche
desiderate. Una possibile opzione è, ad esempio, l'algoritmo
\textit{Linde-Buzo-Gray} (\textit{LBG}).\cite{1094577}
