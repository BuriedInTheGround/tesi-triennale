\section{Approccio feedback-only}

Mostriamo adesso l'ottimalità dell'approccio feedback-only per la
segnalazione del CSI quando un tasso di segnalazione variabile viene
utilizzato. Per semplicità di derivazione, assumiamo che entrambi i canali
\(\bm{h}^\mathrm{(U)}\) e \(\bm{h}^\mathrm{(D)}\) possano essere rappresentati
da alfabeti di dimensione finita molto grande. In particolare, per
\(\bm{h}^\mathrm{(D)}\), consideriamo una quantizzazione tale da garantire
l'MSE desiderato sul canale di DL alla BS. Quindi, assumiamo che la BS intenda
ricostruire perfettamente il canale di DL senza distorsione. Inoltre, imponiamo
un vincolo solamente sul numero totale di bit della segnalazione \(B\), senza
ulteriormente richiedere un bilanciamento tra i tassi di segnalazione in UL e
DL.

\begin{thm}
    \label{thm:feedback-only}

    Per una procedura di segnalazione che utilizza un feedback a tasso
    variabile con media \(b_\mathrm{B}\) e un feedforward a tasso variabile con
    media \(b_\mathrm{F}\), il tasso minimo totale della segnalazione \(B =
    b_\mathrm{B} + b_\mathrm{F}\) è ottenuto con un approccio feedback-only,
    ovvero per \(b_\mathrm{F} = 0\).
\end{thm}

\begin{proof}
    Possiamo vedere la procedura di feedback come una codifica Slepian-Wolf
    (vedi Sezione~\ref{sec:correlated-sources-encoding}) nella quale le due
    sorgenti correlate (i canali di UL e DL) vengono codificate separatamente e
    poi inviate a una destinazione (la BS) che deve ricostruire entrambe. Si
    noti che nel nostro scenario il canale di UL è già noto alla BS, di
    conseguenza questa parte della codifica risulta banale. Ora, in presenza
    di feedforward, lo UT conosce sia la variabile di feedforward
    \(c^\mathrm{(F)}\) che rappresenta la versione quantizzata di
    \(\bm{h}^\mathrm{(U)}\) che il canale di DL \(\bm{h}^\mathrm{(D)}\).
    Pertanto, denotano con \(R_\mathrm{L}\) il tasso medio per la
    rappresentazione locale del canale di UL alla BS, il teorema di
    Slepian-Wolf fornisce
    \begin{alignat}{1}
        b_\mathrm{B} &\ge \entropy{
            \bm{h}^\mathrm{(D)},c^\mathrm{(F)} \given \bm{h}^\mathrm{(U)}
        }, \label{eq:fo-proof-1} \\
        R_\mathrm{L} &\ge \entropy{
            \bm{h}^\mathrm{(U)} \given \bm{h}^\mathrm{(D)},c^\mathrm{(F)}
        }, \label{eq:fo-proof-2} \\
        b_\mathrm{B} + R_\mathrm{L} &\ge \entropy{
            \bm{h}^\mathrm{(D)},\bm{h}^\mathrm{(U)},c^\mathrm{(F)}
        }. \label{eq:fo-proof-3}
    \end{alignat}
    Si noti che, dal momento che non ci sono limitazioni per la
    rappresentazione interna del canale di UL alla BS, la \eqref{eq:fo-proof-2}
    e la \eqref{eq:fo-proof-3} sono trascurabili, o in altre parole, non è
    limitante porre \(R_\mathrm{L} = \infty\); quindi, la sola condizione
    rilevante è
    \begin{equation}
        \label{eq:fo-proof-4}
        b_\mathrm{B} \ge \entropy{
            \bm{h}^\mathrm{(D)},c^\mathrm{(F)} \given \bm{h}^\mathrm{(U)}
        }.
    \end{equation}
    Tuttavia, poiché \(c^\mathrm{(F)}\) è funzione deterministica di
    \(\bm{h}^\mathrm{(U)}\), la \eqref{eq:fo-proof-4} risulta equivalente a
    \begin{equation}
        \label{eq:fo-proof-5}
        b_\mathrm{B} \ge \entropy{
            \bm{h}^\mathrm{(D)} \given \bm{h}^\mathrm{(U)}
        }.
    \end{equation}
    Quindi, abbiamo dimostrato che la segnalazione in feedforward non è utile
    al fine di ridurre i bit di feedback.
\end{proof}

Si noti come, considerando una codifica di sorgenti correlate distribuite
Slepian-Wolf, il risultato ottenuto corrisponde al caso in cui una delle due
sorgenti, anziché essere codificata e inviata al decodificatore, assume il
ruolo di informazione laterale. Infatti, una stima del canale di UL
\(\bm{h}^\mathrm{(U)}\) è già presente alla BS, e l'unica sorgente che
dev'essere codificata (dallo UT) e inviata al decodificatore (la BS) è il
canale di DL \(\bm{h}^\mathrm{(D)}\).

Inoltre, vogliamo sottolineare il fatto che questo risultato è valido in
assenza di condizioni che sbilanciano i tassi di segnalazione in UL e DL,
ovvero quando non ci sono ulteriori limitazioni su \(b_\mathrm{B}\) e
\(b_\mathrm{F}\). Nella pratica, gli svantaggi nel segnalare in UL e DL
potrebbero essere differenti, e soluzioni alternative potrebbero divenire
praticabili.
