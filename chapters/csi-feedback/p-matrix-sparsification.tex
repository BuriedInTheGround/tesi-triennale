\subsection{Sparsificazione della matrice di densità di probabilità discreta
congiunta}

L'approccio proposto risulta tanto più efficace quanto più la matrice di
densità di probabilità discreta congiunta è sparsa. Aumentare le dimensioni
della matrice, ovvero aumentare \(b_\mathrm{B}\) e \(b_\mathrm{U}\), incrementa
la sparsità della matrice, ma avendo un vincolo su \(b_\mathrm{B}\) (necessario
a limitare l'overhead complessivo), e conseguentemente su \(b_\mathrm{U}\) in
base alla prima fase della procedura, è necessario ricorrere a una strategia
differente.

La tecnica proposta consiste nell'indurre deliberatamente sparsità sulla
matrice, ponendo a zero gli elementi della matrice le cui coppie di vettori di
UL-DL presentano una bassa probabilità di verificarsi. A seguito di questo
approccio, quei vettori che prima erano quantizzati da elementi ora posti a
zero devono essere quantizzati utilizzando un'altra coppia di vettori,
ovverosia la seconda più vicina (secondo la metrica considerata). Questa
procedura fornisce certamente una matrice sparsa, a scapito però di un NMSE
maggiore, dal momento che non sempre è utilizzata la miglior coppia di vettori
per la quantizzazione dei canali di UL e DL.
