\subsection*{Notazione}

Presentiamo ora la notazione che verrà utilizzata nel seguito in questo
documento.

Sia \(\C\) l'insieme dei numeri complessi. Indichiamo vettori riga e matrici in
grassetto, rispettivamente con lettere minuscole e maiuscole.  Pertanto,
\(x_i\) indica l'\(i\)-esimo elemento del vettore \(\bm{x}\), mentre
\(A_{i,j}\) denota l'elemento posizionato sull'\(i\)-esima colonna e la
\(j\)-esima riga della matrice \(\bm{A}\). Denotiamo il trasposto della matrice
\(\bm{A}\) con \(\tran{A}\) e il trasposto coniugato con \(\conjtran{A}\).  Il
prodotto interno tra due vettori \(\bm{x},\bm{y} \in \C^{1 \times N}\) è dato
da \(\bm{x}\conjtran{y} \define \sum_i x_i \conj{y_i}\), con \(\conj{y_i}\) il
complesso coniugato di \(y_i\). La norma di un vettore \(\bm{x}\) è definita
come \(\lnorm \bm{x} \rnorm = \lnorm \bm{x} \rnorm_2 \define
\mleft(\bm{x}\conjtran{x}\mright)^{\sfrac{1}{2}}\). Gli insiemi sono denotati
con lettere maiuscole calligrafiche, e \(\abs{\set{A}}\) indica la cardinalità
dell'insieme \(\set{A}\). Infine, denotiamo la probabilità di un evento \(E\)
con \(\prob{E}\), il valore atteso di una variabile casuale \(X\) con
\(\expect{X}\), e l'entropia di \(X\) con \(\entropy{X}\).
