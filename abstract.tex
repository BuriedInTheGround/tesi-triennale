% Define `abstract` environment missing in the `book` class
\newenvironment{abstract}
{\cleardoublepage
\null\vfill
\begin{center}
    \bfseries \abstractname
\end{center}}
{\vfill\null}

\begin{abstract}
    Nelle reti cellulari, la modalità frequency-division duplexing (FDD)
    consente la trasmissione simultanea sul canale di uplink (UL) e downlink
    (DL), e a tale scopo sono adoperate due bande di frequenza separate. Per
    questo motivo la stima del canale di DL è solitamente effettuata dallo user
    terminal (UT) e successivamente inviata alla base station (BS). In questa
    tesi, ho rielaborato una procedura presente in letteratura che permette di
    ottenere una stima del canale di DL alla BS, sfruttando la dipendenza
    parziale tra i canali di UL e DL, e minimizzando il tasso medio della
    segnalazione utile alla procedura stessa. Anzitutto ho riportato il teorema
    presentato in \cite{https://doi.org/10.1002/ett.3628} che, utilizzando il
    teorema di Slepian-Wolf \cite{1055037}, mostra come un approccio
    feedback-only sia sufficiente a minimizzare il tasso della segnalazione per
    effettuare una codifica in assenza di distorsione. Questo risultato porta
    all'elaborazione della procedura sopra citata, riportata in forma estesa in
    questa tesi. Infine, ho citato delle strategie possibili note in
    letteratura per il caso della codifica in presenza di distorsione. Dal
    teorema di Wyner-Ziv \cite{1055508} si ha infatti che la segnalazione di
    feedforward può contribuire a ridurre il tasso di feedback quando si è in
    presenza di distorsione, diversamente dal caso in cui la distorsione è
    assente, ed è quindi necessario fare ricorso a procedure differenti.
\end{abstract}
